\subsection*{Полезные применения производной. Ряд Тейлора.}
\addcontentsline{toc}{subsection}{Полезные применения производной. Ряд Тейлора.}
\paragraph{Полезные применения производной}
\begin{enumerate}
    \item Точка А движется согласно уравнениям $x_1 = 2t, y_1 = t$, а точка В – согласно уравнениям $x_2 = 10-t, y_2 = 2t$ ($x, y$ – в метрах, $t$ – в секундах). Определить расстояние $S$ между этими точками в момент их максимального сближения. Оба движения в одной плоскости,  – координаты точек в прямоугольной системе координат в этой плоскости
    \item К источнику электрической энергии с ЭДС $E$ и внутренним сопротивлением $r$ подключён реостат. Какую наибольшую тепловую мощность можно получить на внешнем участке цепи?
    \item С какой наименьшей скоростью надо бросить мяч, чтобы забросить его на крышу дома высотой $H$ с расстояния $S$ от дома?
    \item $\boldsymbol{*}$ Вписать в круг единичного радиуса прямоугольник наибольшей площади.
    \item $\boldsymbol{*}$ Ядро, летящее со скоростью $v$, распадается на два одинаковыx осколка. Определите максимальный возможный угол $\alpha$ между скоростями одного из осколков и вектором $v$, если при распаде покоящегося ядра осколки имеют скорость $u<v$.
\end{enumerate}
\paragraph{Ряд Тейлора}
\begin{enumerate}
    \item Разложи в ряд Маклорена функции $f(x)$
    \begin{enumerate}
    \begin{minipage}{0.5\textwidth}
        \item $ f(x) = e^{-x} \quad \text{до члена с } x^n$
        \item $f(x) = e^{2x - x^2} \quad \text{до члена с } x^5$
        \item $f(x) = x\sin x \quad \text{до члена с } x^n$
        
    \end{minipage}
     \begin{minipage}{0.5\textwidth}
        \item $ f(x) = e^{-x} \quad \text{до члена с } x^n$
        \item $f(x) = \ln(1 + x) \quad \text{до члена с } x^n$ 
        \item $f(x) = \sqrt{1 + x} \quad \text{до члена с } x^n$
    \end{minipage}
    \end{enumerate}
    \item \textbf{Связать с физикой?}
\end{enumerate}