\subsection*{Производная}
\addcontentsline{toc}{subsection}{Производная}
\begin{enumerate}
    \item По графику функции $y = f(x)$ нарисовать график функции ее производной $y = f^{'}(x)$ (см.вниз\footnote{придумай график сам и нарисуй})
    \item Найдите производные функций по определению, используя предел:
    \begin{enumerate}
    \begin{minipage}{0.5\textwidth}
        \item $y(x) = C = const$
        \item $y(x) = x$
        \item $y(x) = x^2 + 10$ в точке $x = 3$
    \end{minipage}
    \begin{minipage}{0.5\textwidth}
        \item $y(x) = x^3$ в точке $x = 5$
        \item $y(x) = \frac{1}{x}$ в точке $x = 1$
        \item $y(x) = \sqrt{x}$
    \end{minipage}
    \end{enumerate}
    \item Вычислите производные:
    \begin{enumerate}
    \begin{minipage}{0.5\textwidth}
         \item $y(x) = 2 \sqrt[3]{x^2} - \frac{3}{\sqrt{x}}$
        \item $y(x) = \sin x - \cos x$
        \item $y(x) = \tg x - \ctg x$
        \item $y(x) = \cos^2 x \cdot x^2$
    \end{minipage}
    \begin{minipage}{0.5\textwidth}
        \item $y(x) = e^{2x} \sin x$
        \item $y(x) = \frac{x^2 - 2}{\sqrt{x^2 + 1}}$
        \item $y(x) = \sin x \cos x \cdot x $
        \item $y(x) = \sin [ \sin [\sin x]]$
    \end{minipage}
    \end{enumerate}
    \item Написать формулы, задающие координаты точки, равномерно движущейся по окружности, как функции времени. Найти производные этих функций. Что характеризуют эти производные? Как увидеть из полученных формул, что скорость движения направлена по касательной к окружности?
    \item Найти ток в цепи с зарядом $q = A \cos (\alpha e^{-\omega t})$. Когда этот ток максимален?
    \item $\boldsymbol{*}$ Показать, что $\ln x = \lim_{n \rightarrow \infty} n (x^{1/n} - 1)$
    \item $\boldsymbol{*}$ Даны две точки $A$ и $B$ по одну сторону от прямой $l$. Требуется найти на $l$ такую точку $D$, чтобы сумма расстояний от $A$ до $D$ и от $B$ до $D$ была наименьшей.
    \item $\boldsymbol{*}$ Две среды разделены плоской границей. Луч света, идущий из точки, лежащей по одну сторону границы, в точку, лежащую по другую сторону, избирает путь, требующий наименьшего времени. Что это за путь, если скорость движения в указанных средах равна $v_1$ и $v_2$ соответственно?
\end{enumerate}