\subsection*{Семинар 1. Производная.}
\addcontentsline{toc}{subsection}{Производная}
\epigraph{\textsf{Your focus determines your reality}}{\texttt{Qui-Gon Jinn}}

\begin{enumerate}
    \item По графику функции $y = f(x)$ нарисуй график функции ее производной $y = f^{'}(x)$ (см.вниз\footnote{придумай график сам и нарисуй})
    \item Найди производные функций по определению, используя предел:
    \begin{enumerate}
    \begin{minipage}{0.5\textwidth}
        \item $y(x) = C = const$
        \item $y(x) = x$
        \item $y(x) = x^2 + 10$ в точке $x = 3$
        \item $y(x) = \sin x$
    \end{minipage}
    \begin{minipage}{0.5\textwidth}
        \item $y(x) = x^3$ в точке $x = 5$
        \item $y(x) = \frac{1}{x}$ в точке $x = 1$
        \item $y(x) = \sqrt{x}$
        \item $y(x) = \tan x$
    \end{minipage}
    \end{enumerate}
    \item Вычисли производные, но уже не используя предел:
    \begin{enumerate}
    \begin{minipage}{0.5\textwidth}
        \item $y(x) = 2 \sqrt[3]{x^2} - \frac{3}{\sqrt{x}}$
        \item $y(x) = \sin x - \cos x$
        \item $y(x) = \tg x - \ctg x$
        \item $y(x) = \cos^2 x \cdot x^2$
    \end{minipage}
    \begin{minipage}{0.5\textwidth}
        \item $y(x) = e^{2x} \sin x$
        \item $y(x) = \frac{x^2 - 2}{\sqrt{x^2 + 1}}$
        \item $y(x) = \sin x \cos x \cdot x $
        \item $y(x) = \sin [ \sin [\sin x]]$
    \end{minipage}
    \end{enumerate}
    \item Найди производную функций $f(x) = \cosh x = \frac{e^x + e^{-x}}{2}$ и $g(x) = \sinh x = \frac{e^x - e^{-x}}{2}$ (cм. вниз\footnote{такие функции называются \textbf{гиперболическими}}). Каким соотношением связаны $f(x)$ и $g(x)$? Какое значение принимает $f(0)$ и $g(0)$? Какое значение принимают производные этих функций в нуле?
    \item Напиши формулы, задающие координаты точки, равномерно движущейся по окружности, как функции времени. Найди производные этих функций. Что характеризуют эти производные? Как увидеть из полученных формул, что скорость движения направлена по касательной к окружности?
    \item Найди следующее:
    \begin{enumerate}
        \item Cилу тока $I$, если $q = A e^{-\gamma t} \cos t$
        \item Силу $F_x$, если $x(t) = x_0 \cos \omega t$
        \item Силу $F$, если импульс равен $p = \frac{mv}{\sqrt{1 - \frac{v^2}{c^2}}}$
        \item $\boldsymbol{*}$ Найди силы $F_x, F_y, F_z$, если $W = \frac{m(x^2 + e^2y + z^3)}{2}$
    \end{enumerate}
    \item $\boldsymbol{*}$ Показажи, что $\displaystyle \ln x = \lim_{n \rightarrow \infty} n (x^{1/n} - 1)$ (см. вниз \footnote{этот предел поможет вам разгадать загадку задачи с первого занятия})
    \item $\boldsymbol{**}$ Найди предел $\displaystyle \lim_{n \rightarrow \infty} \cos \frac{x}{2} \cos \frac{x}{4} \dots \cos \frac{x}{2^n}$ (см. вниз \footnote{эта задача была на вступительных экзаменах в ШАД Яндекса})
    \item $\boldsymbol{***}$ Камень бросили со скоростью $v_0$ под углом $\alpha$ к горизонту. Найди модуль радиус-вектора $r(t)$ (вектор смотрит на траекторию из начала полета). Какова скорость изменения $r(t)$?
\end{enumerate}