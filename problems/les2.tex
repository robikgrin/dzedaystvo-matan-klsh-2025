\subsection*{Производная}
\addcontentsline{toc}{subsection}{Производная}
\begin{enumerate}
    \item По графику функции $y = f(x)$ нарисуй график функции ее производной $y = f^{'}(x)$ (см.вниз\footnote{придумай график сам и нарисуй})
    \item Найди производные функций по определению, используя предел:
    \begin{enumerate}
    \begin{minipage}{0.5\textwidth}
        \item $y(x) = C = const$
        \item $y(x) = x$
        \item $y(x) = x^2 + 10$ в точке $x = 3$
    \end{minipage}
    \begin{minipage}{0.5\textwidth}
        \item $y(x) = x^3$ в точке $x = 5$
        \item $y(x) = \frac{1}{x}$ в точке $x = 1$
        \item $y(x) = \sqrt{x}$
    \end{minipage}
    \end{enumerate}
    \item Вычисли производные:
    \begin{enumerate}
    \begin{minipage}{0.5\textwidth}
         \item $y(x) = 2 \sqrt[3]{x^2} - \frac{3}{\sqrt{x}}$
        \item $y(x) = \sin x - \cos x$
        \item $y(x) = \tg x - \ctg x$
        \item $y(x) = \cos^2 x \cdot x^2$
    \end{minipage}
    \begin{minipage}{0.5\textwidth}
        \item $y(x) = e^{2x} \sin x$
        \item $y(x) = \frac{x^2 - 2}{\sqrt{x^2 + 1}}$
        \item $y(x) = \sin x \cos x \cdot x $
        \item $y(x) = \sin [ \sin [\sin x]]$
    \end{minipage}
    \end{enumerate}
    \item Найди производную функций $f(x) = \cosh x = \frac{e^x + e^{-x}}{2}$ и $g(x) = \sinh x = \frac{e^x - e^{-x}}{2}$ (cм. вниз\footnote{такие функции называются \textbf{гиперболическими}}). Каким соотношением связаны $f(x)$ и $g(x)$? Какое значение принимает $f(0)$ и $g(0)$? Какое значение принимают производные этих функций в нуле?
    \item Напиши формулы, задающие координаты точки, равномерно движущейся по окружности, как функции времени. Найти производные этих функций. Что характеризуют эти производные? Как увидеть из полученных формул, что скорость движения направлена по касательной к окружности?
    \item Найди ток в цепи с зарядом $q = A \cos (\alpha e^{-\omega t})$. Когда этот ток максимален?
    \item $\boldsymbol{*}$ Показажи, что $\displaystyle\ln x = \lim_{n \rightarrow \infty} n (x^{1/n} - 1)$
\end{enumerate}