\section*{Советы для семинаристов}
\addcontentsline{toc}{section}{Советы для семинаристов}

Часть этого руководства взята из плана к курсу Никиты Астраханцева (2017 год). Сами же советы были сформулированы Артемом Абановым. Были добавлены некоторые дополнения.
\begin{itemize}
    \item В начале каждого семинара делать наставления для школьников (2-3 минуты). Рассказать какую-нибудь байку или анекдот. 
    \item Школьники делятся на разные команды (столы) по их уровню знаний. Нужно сделать так, чтобы до школьника было близко идти, чтобы успеть поработать с каждым за вашим столом.
    \item Готовтесь к каждому семинару!!! Подумайте над задачами, которые вы можете дать конкретному школьнику, помимо заранее подготовленных. Сделайте это таким образом, чтобы вы их быстро могли вспомнить: сделайте распечатку, перепишите в тетрадь.
    \item Ваша задача - подобрать задачу уровня, немногим выше уровня школьника, чтобы ему/ей было комфортно, удобно и процесс был продуктивным.
    \item Если школьник решил задачу, то следует сделать следующее: 
    \begin{enumerate}
        \item Похвалить
        \item Проверить размерность/знак/ответ
        \item Обсудить предельные случаи. Поговорить о физике этих предельных случаев.
    \end{enumerate}
    \item Если задача у школьника вызывает затруднения, то стоит его подбодрить и помочь коротким советом.
    \item Если затруднения продолжаются, то стоит дать другую задачу, которая подведет к нерешенной задаче.
    \item Если видите, что школьники устали - рассказать байку/анекдот в тему занятия. 
    \item Всегда старайтесь добиться от школьника физического смысла, заложенного в формулах. Формулы - отражение физики задачи! 
\end{itemize}